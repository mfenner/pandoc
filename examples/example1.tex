\emph{Article-level metrics (ALMs) provide a wide range of metrics about
the uptake of an individual journal article by the scientific community
after publication. They include citations, usage statistics, discussions
in online comments and social media, social bookmarking, and
recommendations. In this essay, we describe why article-level metrics
are an important extension of traditional citation-based journal metrics
and provide a number of example from ALM data collected for PLOS
Biology.}

\begin{quote}
This is an open-access article distributed under the terms of the
Creative Commons Attribution License, authored by me and
\href{http://dx.doi.org/10.1371/journal.pbio.1001687}{originally
published Oct 22, 2013 in PLOS Biology}.
\end{quote}

The scientific impact of a particular piece of research is reflected in
how this work is taken up by the scientific community. The first
systematic approach that was used to assess impact, based on the
technology available at the time, was to track citations and aggregate
them by journal. This strategy is not only no longer necessary --- since
now we can easily track citations for individual articles --- but also,
and more importantly, journal-based metrics are now considered a poor
performance measure for individual articles {[}@Campbell2008;
@Glanzel2013{]}. One major problem with journal-based metrics is the
variation in citations per article, which means that a small percentage
of articles can skew, and are responsible for, the majority of the
journal-based citation impact factor, as shown by Campbell
{[}-@Campbell2008{]} for the 2004 \emph{Nature} Journal Impact Factor.
\textbf{Figure 1} further illustrates this point, showing the wide
distribution of citation counts between \emph{PLOS Biology} research
articles published in 2010. \emph{PLOS Biology} research articles
published in 2010 have been cited a median 19 times to date in Scopus,
but 10\% of them have been cited 50 or more times, and two articles
{[}@Narendra:2010fw; @Dickson:2010ix{]} more than 300 times. \emph{PLOS
Biology} metrics are used as examples throughout this essay, and the
dataset is available in the supporting information (\textbf{Data S1}).
Similar data are available for an increasing number of other
publications and organizations.

\begin{Shaded}
\begin{Highlighting}[]
\CommentTok{# code for figure 1: density plots for citation counts for PLOS Biology}
\CommentTok{# articles published in 2010}

\CommentTok{# load May 20, 2013 ALM report}
\NormalTok{alm <-}\StringTok{ }\KeywordTok{read.csv}\NormalTok{(}\StringTok{"../data/alm_report_plos_biology_2013-05-20.csv"}\NormalTok{, }\DataTypeTok{stringsAsFactors =} \OtherTok{FALSE}\NormalTok{)}

\CommentTok{# only look at research articles}
\NormalTok{alm <-}\StringTok{ }\KeywordTok{subset}\NormalTok{(alm, alm$article_type ==}\StringTok{ "Research Article"}\NormalTok{)}

\CommentTok{# only look at papers published in 2010}
\NormalTok{alm$publication_date <-}\StringTok{ }\KeywordTok{as.Date}\NormalTok{(alm$publication_date)}
\NormalTok{alm <-}\StringTok{ }\KeywordTok{subset}\NormalTok{(alm, alm$publication_date >}\StringTok{ "2010-01-01"} \NormalTok{&}\StringTok{ }\NormalTok{alm$publication_date <=}
\StringTok{    "2010-12-31"}\NormalTok{)}

\CommentTok{# labels}
\NormalTok{colnames <-}\StringTok{ }\KeywordTok{dimnames}\NormalTok{(alm)[[}\DecValTok{2}\NormalTok{]]}
\NormalTok{plos.color <-}\StringTok{ "#1ebd21"}
\NormalTok{plos.source <-}\StringTok{ "scopus"}

\NormalTok{plos.xlab <-}\StringTok{ "Scopus Citations"}
\NormalTok{plos.ylab <-}\StringTok{ "Probability"}

\NormalTok{quantile <-}\StringTok{ }\KeywordTok{quantile}\NormalTok{(alm[, plos.source], }\KeywordTok{c}\NormalTok{(}\FloatTok{0.1}\NormalTok{, }\FloatTok{0.5}\NormalTok{, }\FloatTok{0.9}\NormalTok{), }\DataTypeTok{na.rm =} \OtherTok{TRUE}\NormalTok{)}

\CommentTok{# plot the chart}
\NormalTok{opar <-}\StringTok{ }\KeywordTok{par}\NormalTok{(}\DataTypeTok{mai =} \KeywordTok{c}\NormalTok{(}\FloatTok{0.5}\NormalTok{, }\FloatTok{0.75}\NormalTok{, }\FloatTok{0.5}\NormalTok{, }\FloatTok{0.5}\NormalTok{), }\DataTypeTok{omi =} \KeywordTok{c}\NormalTok{(}\FloatTok{0.25}\NormalTok{, }\FloatTok{0.1}\NormalTok{, }\FloatTok{0.25}\NormalTok{, }\FloatTok{0.1}\NormalTok{), }\DataTypeTok{mgp =} \KeywordTok{c}\NormalTok{(}\DecValTok{3}\NormalTok{,}
    \FloatTok{0.5}\NormalTok{, }\FloatTok{0.5}\NormalTok{), }\DataTypeTok{fg =} \StringTok{"black"}\NormalTok{, }\DataTypeTok{cex.main =} \DecValTok{2}\NormalTok{, }\DataTypeTok{cex.lab =} \FloatTok{1.5}\NormalTok{, }\DataTypeTok{col =} \NormalTok{plos.color,}
    \DataTypeTok{col.main =} \NormalTok{plos.color, }\DataTypeTok{col.lab =} \NormalTok{plos.color, }\DataTypeTok{xaxs =} \StringTok{"i"}\NormalTok{, }\DataTypeTok{yaxs =} \StringTok{"i"}\NormalTok{)}

\NormalTok{d <-}\StringTok{ }\KeywordTok{density}\NormalTok{(alm[, plos.source], }\DataTypeTok{from =} \DecValTok{0}\NormalTok{, }\DataTypeTok{to =} \DecValTok{100}\NormalTok{)}
\NormalTok{d$x <-}\StringTok{ }\KeywordTok{append}\NormalTok{(d$x, }\DecValTok{0}\NormalTok{)}
\NormalTok{d$y <-}\StringTok{ }\KeywordTok{append}\NormalTok{(d$y, }\DecValTok{0}\NormalTok{)}
\KeywordTok{plot}\NormalTok{(d, }\DataTypeTok{type =} \StringTok{"n"}\NormalTok{, }\DataTypeTok{main =} \OtherTok{NA}\NormalTok{, }\DataTypeTok{xlab =} \OtherTok{NA}\NormalTok{, }\DataTypeTok{ylab =} \OtherTok{NA}\NormalTok{, }\DataTypeTok{xlim =} \KeywordTok{c}\NormalTok{(}\DecValTok{0}\NormalTok{, }\DecValTok{100}\NormalTok{), }\DataTypeTok{frame.plot =} \OtherTok{FALSE}\NormalTok{)}
\KeywordTok{polygon}\NormalTok{(d, }\DataTypeTok{col =} \NormalTok{plos.color, }\DataTypeTok{border =} \OtherTok{NA}\NormalTok{)}
\KeywordTok{mtext}\NormalTok{(plos.xlab, }\DataTypeTok{side =} \DecValTok{1}\NormalTok{, }\DataTypeTok{col =} \NormalTok{plos.color, }\DataTypeTok{cex =} \FloatTok{1.25}\NormalTok{, }\DataTypeTok{outer =} \OtherTok{TRUE}\NormalTok{, }\DataTypeTok{adj =} \DecValTok{1}\NormalTok{,}
    \DataTypeTok{at =} \DecValTok{1}\NormalTok{)}
\KeywordTok{mtext}\NormalTok{(plos.ylab, }\DataTypeTok{side =} \DecValTok{2}\NormalTok{, }\DataTypeTok{col =} \NormalTok{plos.color, }\DataTypeTok{cex =} \FloatTok{1.25}\NormalTok{, }\DataTypeTok{outer =} \OtherTok{TRUE}\NormalTok{, }\DataTypeTok{adj =} \DecValTok{0}\NormalTok{,}
    \DataTypeTok{at =} \DecValTok{1}\NormalTok{, }\DataTypeTok{las =} \DecValTok{1}\NormalTok{)}

\KeywordTok{par}\NormalTok{(opar)}
\end{Highlighting}
\end{Shaded}

\begin{figure}[htbp]
\centering
\includegraphics{/images/2013-12-11_figure_1.svg}
\caption{\textbf{Figure 1. Citation counts for PLOS Biology articles
published in 2010.} Scopus citation counts plotted as a probability
distribution for all 197 \emph{PLOS Biology} research articles published
in 2010. Data collected May 20, 2013. Median 19 citations; 10\% of
papers have at least 50 citations.}
\end{figure}

Scientific impact is a multi-dimensional construct that can not be
adequately measured by any single indicator {[}@Glanzel2013;
@bollenEtal2009; @Schekman2013{]}. To this end, PLOS has collected and
displayed a variety of metrics for all its articles since 2009. The
array of different categorised article-level metrics (ALMs) used and
provided by PLOS as of August 2013 are shown in \textbf{Figure 2}. In
addition to citations and usage statistics, i.e., how often an article
has been viewed and downloaded, PLOS also collects metrics about: how
often an article has been saved in online reference managers, such as
Mendeley; how often an article has been discussed in its comments
section online, and also in science blogs or in social media; and how
often an article has been recommended by other scientists. These
additional metrics provide valuable information that we would miss if we
only consider citations. Two important shortcomings of citation-based
metrics are that (1) they take years to accumulate and (2) citation
analysis is not always the best indicator of impact in more practical
fields, such as clinical medicine {[}@VanEck2013{]}. Usage statistics
often better reflect the impact of work in more practical fields, and
they also sometimes better highlight articles of general interest (for
example, the 2006 \emph{PLOS Biology} article on the citation advantage
of Open Access articles {[}@Eysenbach2006{]}, one of the 10 most-viewed
articles published in \emph{PLOS Biology}).

\begin{figure}[htbp]
\centering
\includegraphics{/images/2013-12-11_figure_2.png}
\caption{\textbf{Figure 2. Article-level metrics used by PLOS in August
2013 and their categories.} Taken from {[}@Lin2013{]} with permission by
the authors.}
\end{figure}

A bubble chart showing all 2010 \emph{PLOS Biology} articles
(\textbf{Figure 3}) gives a good overview of the year's views and
citations, plus it shows the influence that the article type (as
indicated by dot color) has on an article's performance as measured by
these metrics. The weekly \emph{PLOS Biology} publication schedule is
reflected in this figure, with articles published on the same day
present in a vertical line. \textbf{Figure 3} also shows that the two
most highly cited 2010 \emph{PLOS Biology} research articles are also
among the most viewed (indicated by the red arrows), but overall there
isn't a strong correlation between citations and views. The most-viewed
article published in 2010 in \emph{PLOS Biology} is an essay on
Darwinian selection in robots {[}@Floreano2010{]}. Detailed usage
statistics also allow speculatulation about the different ways that
readers access and make use of published literature; some articles are
browsed or read online due to general interest while others that are
downloaded (and perhaps also printed) may reflect the reader's intention
to look at the data and results in detail and to return to the article
more than once.

\begin{Shaded}
\begin{Highlighting}[]
\CommentTok{# code for figure 3: Bubblechart views vs. citations for PLOS Biology}
\CommentTok{# articles published in 2010.}

\CommentTok{# Load required libraries}
\KeywordTok{library}\NormalTok{(plyr)}

\CommentTok{# load May 20, 2013 ALM report}
\NormalTok{alm <-}\StringTok{ }\KeywordTok{read.csv}\NormalTok{(}\StringTok{"../data/alm_report_plos_biology_2013-05-20.csv"}\NormalTok{, }\DataTypeTok{stringsAsFactors =} \OtherTok{FALSE}\NormalTok{,}
    \DataTypeTok{na.strings =} \KeywordTok{c}\NormalTok{(}\StringTok{"0"}\NormalTok{))}

\CommentTok{# only look at papers published in 2010}
\NormalTok{alm$publication_date <-}\StringTok{ }\KeywordTok{as.Date}\NormalTok{(alm$publication_date)}
\NormalTok{alm <-}\StringTok{ }\KeywordTok{subset}\NormalTok{(alm, alm$publication_date >}\StringTok{ "2010-01-01"} \NormalTok{&}\StringTok{ }\NormalTok{alm$publication_date <=}
\StringTok{    "2010-12-31"}\NormalTok{)}

\CommentTok{# make sure counter values are numbers}
\NormalTok{alm$counter_html <-}\StringTok{ }\KeywordTok{as.numeric}\NormalTok{(alm$counter_html)}

\CommentTok{# lump all papers together that are not research articles}
\NormalTok{reassignType <-}\StringTok{ }\NormalTok{function(x) if (x ==}\StringTok{ "Research Article"}\NormalTok{) }\DecValTok{1} \NormalTok{else }\DecValTok{0}
\NormalTok{alm$article_group <-}\StringTok{ }\KeywordTok{aaply}\NormalTok{(alm$article_type, }\DecValTok{1}\NormalTok{, reassignType)}

\CommentTok{# calculate article age in months}
\NormalTok{alm$age_in_months <-}\StringTok{ }\NormalTok{(}\KeywordTok{Sys.Date}\NormalTok{() -}\StringTok{ }\NormalTok{alm$publication_date)/}\FloatTok{365.25} \NormalTok{*}\StringTok{ }\DecValTok{12}
\NormalTok{start_age_in_months <-}\StringTok{ }\KeywordTok{floor}\NormalTok{(}\KeywordTok{as.numeric}\NormalTok{(}\KeywordTok{Sys.Date}\NormalTok{() -}\StringTok{ }\KeywordTok{as.Date}\NormalTok{(}\KeywordTok{strptime}\NormalTok{(}\StringTok{"2010-12-31"}\NormalTok{,}
    \DataTypeTok{format =} \StringTok{"%Y-%m-%d"}\NormalTok{)))/}\FloatTok{365.25} \NormalTok{*}\StringTok{ }\DecValTok{12}\NormalTok{)}

\CommentTok{# chart variables}
\NormalTok{x <-}\StringTok{ }\NormalTok{alm$age_in_months}
\NormalTok{y <-}\StringTok{ }\NormalTok{alm$counter}
\NormalTok{z <-}\StringTok{ }\NormalTok{alm$scopus}

\NormalTok{xlab <-}\StringTok{ "Age in Months"}
\NormalTok{ylab <-}\StringTok{ "Total Views"}

\NormalTok{labels <-}\StringTok{ }\NormalTok{alm$article_group}
\NormalTok{col.main <-}\StringTok{ "#1ebd21"}
\NormalTok{col <-}\StringTok{ "#666358"}

\CommentTok{# calculate bubble diameter}
\NormalTok{z <-}\StringTok{ }\KeywordTok{sqrt}\NormalTok{(z/pi)}

\CommentTok{# calculate bubble color}
\NormalTok{getColor <-}\StringTok{ }\NormalTok{function(x) }\KeywordTok{c}\NormalTok{(}\StringTok{"#c9c9c7"}\NormalTok{, }\StringTok{"#1ebd21"}\NormalTok{)[x +}\StringTok{ }\DecValTok{1}\NormalTok{]}
\NormalTok{colors <-}\StringTok{ }\KeywordTok{aaply}\NormalTok{(labels, }\DecValTok{1}\NormalTok{, getColor)}

\CommentTok{# plot the chart}
\NormalTok{opar <-}\StringTok{ }\KeywordTok{par}\NormalTok{(}\DataTypeTok{mai =} \KeywordTok{c}\NormalTok{(}\FloatTok{0.5}\NormalTok{, }\FloatTok{0.75}\NormalTok{, }\FloatTok{0.5}\NormalTok{, }\FloatTok{0.5}\NormalTok{), }\DataTypeTok{omi =} \KeywordTok{c}\NormalTok{(}\FloatTok{0.25}\NormalTok{, }\FloatTok{0.1}\NormalTok{, }\FloatTok{0.25}\NormalTok{, }\FloatTok{0.1}\NormalTok{), }\DataTypeTok{mgp =} \KeywordTok{c}\NormalTok{(}\DecValTok{3}\NormalTok{,}
    \FloatTok{0.5}\NormalTok{, }\FloatTok{0.5}\NormalTok{), }\DataTypeTok{fg =} \StringTok{"black"}\NormalTok{, }\DataTypeTok{cex =} \DecValTok{1}\NormalTok{, }\DataTypeTok{cex.main =} \DecValTok{2}\NormalTok{, }\DataTypeTok{cex.lab =} \FloatTok{1.5}\NormalTok{, }\DataTypeTok{col =} \StringTok{"white"}\NormalTok{,}
    \DataTypeTok{col.main =} \NormalTok{col.main, }\DataTypeTok{col.lab =} \NormalTok{col)}

\KeywordTok{plot}\NormalTok{(x, y, }\DataTypeTok{type =} \StringTok{"n"}\NormalTok{, }\DataTypeTok{xlim =} \KeywordTok{c}\NormalTok{(start_age_in_months, start_age_in_months +}\StringTok{ }\DecValTok{13}\NormalTok{),}
    \DataTypeTok{ylim =} \KeywordTok{c}\NormalTok{(}\DecValTok{0}\NormalTok{, }\DecValTok{60000}\NormalTok{), }\DataTypeTok{xlab =} \OtherTok{NA}\NormalTok{, }\DataTypeTok{ylab =} \OtherTok{NA}\NormalTok{, }\DataTypeTok{las =} \DecValTok{1}\NormalTok{)}
\KeywordTok{symbols}\NormalTok{(x, y, }\DataTypeTok{circles =} \NormalTok{z, }\DataTypeTok{inches =} \KeywordTok{exp}\NormalTok{(}\FloatTok{1.3}\NormalTok{)/}\DecValTok{15}\NormalTok{, }\DataTypeTok{bg =} \NormalTok{colors, }\DataTypeTok{xlim =} \KeywordTok{c}\NormalTok{(start_age_in_months,}
    \NormalTok{start_age_in_months +}\StringTok{ }\DecValTok{13}\NormalTok{), }\DataTypeTok{ylim =} \KeywordTok{c}\NormalTok{(}\DecValTok{0}\NormalTok{, ymax), }\DataTypeTok{xlab =} \OtherTok{NA}\NormalTok{, }\DataTypeTok{ylab =} \OtherTok{NA}\NormalTok{, }\DataTypeTok{las =} \DecValTok{1}\NormalTok{,}
    \DataTypeTok{add =} \OtherTok{TRUE}\NormalTok{)}
\KeywordTok{mtext}\NormalTok{(xlab, }\DataTypeTok{side =} \DecValTok{1}\NormalTok{, }\DataTypeTok{col =} \NormalTok{col.main, }\DataTypeTok{cex =} \FloatTok{1.25}\NormalTok{, }\DataTypeTok{outer =} \OtherTok{TRUE}\NormalTok{, }\DataTypeTok{adj =} \DecValTok{1}\NormalTok{, }\DataTypeTok{at =} \DecValTok{1}\NormalTok{)}
\KeywordTok{mtext}\NormalTok{(ylab, }\DataTypeTok{side =} \DecValTok{2}\NormalTok{, }\DataTypeTok{col =} \NormalTok{col.main, }\DataTypeTok{cex =} \FloatTok{1.25}\NormalTok{, }\DataTypeTok{outer =} \OtherTok{TRUE}\NormalTok{, }\DataTypeTok{adj =} \DecValTok{0}\NormalTok{, }\DataTypeTok{at =} \DecValTok{1}\NormalTok{,}
    \DataTypeTok{las =} \DecValTok{1}\NormalTok{)}

\KeywordTok{par}\NormalTok{(opar)}
\end{Highlighting}
\end{Shaded}

\begin{figure}[htbp]
\centering
\includegraphics{/images/2013-12-11_figure_3.svg}
\caption{\textbf{Figure 3. Views vs.~citations for PLOS Biology articles
published in 2010.} All 304 \emph{PLOS Biology} articles published in
2010. Bubble size correlates with number of Scopus citations. Research
articles are labeled green; all other articles are grey. Red arrows
indicate the two most highly cited papers. Data collected May 20, 2013.}
\end{figure}

When readers first see an interesting article, their response is often
to view or download it. By contrast, a citation may be one of the last
outcomes of their interest, occuring only about 1 in 300 times a PLOS
paper is viewed online. A lot of things happen in between these
potential responses, ranging from discussions in comments, social media,
and blogs, to bookmarking, to linking from websites. These activities
are usually subsumed under the term ``altmetrics,'' and their variety
can be overwhelming. Therefore, it helps to group them together into
categories, and several organizations, including PLOS, are using the
category labels of Viewed, Cited, Saved, Discussed, and Recommended
(\textbf{Figures 2 and 4}, see also {[}@Lin2013{]}).

\begin{Shaded}
\begin{Highlighting}[]
\CommentTok{# code for figure 4: bar plot for Article-level metrics for PLOS Biology}

\CommentTok{# Load required libraries}
\KeywordTok{library}\NormalTok{(reshape2)}

\CommentTok{# load May 20, 2013 ALM report}
\NormalTok{alm <-}\StringTok{ }\KeywordTok{read.csv}\NormalTok{(}\StringTok{"../data/alm_report_plos_biology_2013-05-20.csv"}\NormalTok{, }\DataTypeTok{stringsAsFactors =} \OtherTok{FALSE}\NormalTok{,}
    \DataTypeTok{na.strings =} \KeywordTok{c}\NormalTok{(}\DecValTok{0}\NormalTok{, }\StringTok{"0"}\NormalTok{))}

\CommentTok{# only look at research articles}
\NormalTok{alm <-}\StringTok{ }\KeywordTok{subset}\NormalTok{(alm, alm$article_type ==}\StringTok{ "Research Article"}\NormalTok{)}

\CommentTok{# make sure columns are in the right format}
\NormalTok{alm$counter_html <-}\StringTok{ }\KeywordTok{as.numeric}\NormalTok{(alm$counter_html)}
\NormalTok{alm$mendeley <-}\StringTok{ }\KeywordTok{as.numeric}\NormalTok{(alm$mendeley)}

\CommentTok{# options}
\NormalTok{plos.color <-}\StringTok{ "#1ebd21"}
\NormalTok{plos.colors <-}\StringTok{ }\KeywordTok{c}\NormalTok{(}\StringTok{"#a17f78"}\NormalTok{, }\StringTok{"#ad9a27"}\NormalTok{, }\StringTok{"#ad9a27"}\NormalTok{, }\StringTok{"#ad9a27"}\NormalTok{, }\StringTok{"#ad9a27"}\NormalTok{, }\StringTok{"#ad9a27"}\NormalTok{,}
    \StringTok{"#dcebdd"}\NormalTok{, }\StringTok{"#dcebdd"}\NormalTok{, }\StringTok{"#789aa1"}\NormalTok{, }\StringTok{"#789aa1"}\NormalTok{, }\StringTok{"#789aa1"}\NormalTok{, }\StringTok{"#304345"}\NormalTok{, }\StringTok{"#304345"}\NormalTok{)}

\CommentTok{# use subset of columns}
\NormalTok{alm <-}\StringTok{ }\KeywordTok{subset}\NormalTok{(alm, }\DataTypeTok{select =} \KeywordTok{c}\NormalTok{(}\StringTok{"f1000"}\NormalTok{, }\StringTok{"wikipedia"}\NormalTok{, }\StringTok{"researchblogging"}\NormalTok{, }\StringTok{"comments"}\NormalTok{,}
    \StringTok{"facebook"}\NormalTok{, }\StringTok{"twitter"}\NormalTok{, }\StringTok{"citeulike"}\NormalTok{, }\StringTok{"mendeley"}\NormalTok{, }\StringTok{"pubmed"}\NormalTok{, }\StringTok{"crossref"}\NormalTok{, }\StringTok{"scopus"}\NormalTok{,}
    \StringTok{"pmc_html"}\NormalTok{, }\StringTok{"counter_html"}\NormalTok{))}

\CommentTok{# calculate percentage of values that are not missing (i.e. have a count of}
\CommentTok{# at least 1)}
\NormalTok{colSums <-}\StringTok{ }\KeywordTok{colSums}\NormalTok{(!}\KeywordTok{is.na}\NormalTok{(alm)) *}\StringTok{ }\DecValTok{100}\NormalTok{/}\KeywordTok{length}\NormalTok{(alm$counter_html)}
\NormalTok{exactSums <-}\StringTok{ }\KeywordTok{sum}\NormalTok{(}\KeywordTok{as.numeric}\NormalTok{(alm$pmc_html), }\DataTypeTok{na.rm =} \OtherTok{TRUE}\NormalTok{)}

\CommentTok{# plot the chart}
\NormalTok{opar <-}\StringTok{ }\KeywordTok{par}\NormalTok{(}\DataTypeTok{mar =} \KeywordTok{c}\NormalTok{(}\FloatTok{0.1}\NormalTok{, }\FloatTok{7.25}\NormalTok{, }\FloatTok{0.1}\NormalTok{, }\FloatTok{0.1}\NormalTok{) +}\StringTok{ }\FloatTok{0.1}\NormalTok{, }\DataTypeTok{omi =} \KeywordTok{c}\NormalTok{(}\FloatTok{0.1}\NormalTok{, }\FloatTok{0.25}\NormalTok{, }\FloatTok{0.1}\NormalTok{, }\FloatTok{0.1}\NormalTok{),}
    \DataTypeTok{col.main =} \NormalTok{plos.color)}

\NormalTok{plos.names <-}\StringTok{ }\KeywordTok{c}\NormalTok{(}\StringTok{"F1000Prime"}\NormalTok{, }\StringTok{"Wikipedia"}\NormalTok{, }\StringTok{"Research Blogging"}\NormalTok{, }\StringTok{"PLOS Comments"}\NormalTok{,}
    \StringTok{"Facebook"}\NormalTok{, }\StringTok{"Twitter"}\NormalTok{, }\StringTok{"CiteULike"}\NormalTok{, }\StringTok{"Mendeley"}\NormalTok{, }\StringTok{"PubMed Citations"}\NormalTok{, }\StringTok{"CrossRef"}\NormalTok{,}
    \StringTok{"Scopus"}\NormalTok{, }\StringTok{"PMC HTML Views"}\NormalTok{, }\StringTok{"PLOS HTML Views"}\NormalTok{)}
\NormalTok{y <-}\StringTok{ }\KeywordTok{barplot}\NormalTok{(colSums, }\DataTypeTok{horiz =} \OtherTok{TRUE}\NormalTok{, }\DataTypeTok{col =} \NormalTok{plos.colors, }\DataTypeTok{border =} \OtherTok{NA}\NormalTok{, }\DataTypeTok{xlab =} \NormalTok{plos.names,}
    \DataTypeTok{xlim =} \KeywordTok{c}\NormalTok{(}\DecValTok{0}\NormalTok{, }\DecValTok{120}\NormalTok{), }\DataTypeTok{axes =} \OtherTok{FALSE}\NormalTok{, }\DataTypeTok{names.arg =} \NormalTok{plos.names, }\DataTypeTok{las =} \DecValTok{1}\NormalTok{, }\DataTypeTok{adj =} \DecValTok{0}\NormalTok{)}
\KeywordTok{text}\NormalTok{(colSums +}\StringTok{ }\DecValTok{6}\NormalTok{, y, }\DataTypeTok{labels =} \KeywordTok{sprintf}\NormalTok{(}\StringTok{"%1.0f%%"}\NormalTok{, colSums))}

\KeywordTok{par}\NormalTok{(opar)}
\end{Highlighting}
\end{Shaded}

\begin{figure}[htbp]
\centering
\includegraphics{/images/2013-12-11_figure_4.svg}
\caption{\textbf{Figure 4. Article-level metrics for PLOS Biology.}
Proportion of all 1,706 \emph{PLOS Biology} research articles published
up to May 20, 2013 mentioned by particular article-level metrics source.
Colors indicate categories (Viewed, Cited, Saved, Discussed,
Recommended), as used on the PLOS website.}
\end{figure}

All \emph{PLOS Biology} articles are viewed and downloaded, and almost
all of them (all research articles and nearly all front matter) will be
cited sooner or later. Almost all of them will also be bookmarked in
online reference managers, such as Mendeley, but the percentage of
articles that are discussed online is much smaller. Some of these
percentages are time dependent; the use of social media discussion
platforms, such as Twitter and Facebook for example, has increased in
recent years (93\% of \emph{PLOS Biology} research articles published
since June 2012 have been discussed on Twitter, and 63\% mentioned on
Facebook). These are the locations where most of the online discussion
around published articles currently seems to take place; the percentage
of papers with comments on the PLOS website or that have science blog
posts written about them is much smaller. Not all of this online
discussion is about research articles, and perhaps, not surprisingly,
the most-tweeted PLOS article overall (with more than 1,100 tweets) is a
\emph{PLOS Biology} perspective on the use of social media for
scientists {[}@Bik2013{]}.

Some metrics are not so much indicators of a broad online discussion,
but rather focus on highlighting articles of particular interest. For
example, science blogs allow a more detailed discussion of an article as
compared to comments or tweets, and journals themselves sometimes choose
to highlight a paper on their own blogs, allowing for a more digestible
explanation of the science for the non-expert reader {[}@Fausto2012{]}.
Coverage by other bloggers also serves the same purpose; a good example
of this is one recent post on the OpenHelix Blog {[}@Video2012{]} that
contains video footage of the second author of a 2010 \emph{PLOS
Biology} article {[}@Dalloul2010{]} discussing the turkey genome.

F1000Prime, a commercial service of recommendations by expert
scientists, was added to the PLOS Article-Level Metrics in August 2013.
We now highlight on the PLOS website when any articles have received at
least one recommendation within F1000Prime. We also monitor when an
article has been cited within the widely used modern-day online
encyclopedia, Wikipedia. A good example of the latter is the Tasmanian
devil Wikipedia page {[}@Tasmanian2013{]} that links to a \emph{PLOS
Biology} research article published in 2010 {[}@Nilsson2010{]}. While a
F1000Prime recommendation is a strong endorsement from peer(s) in the
scientific community, being included in a Wikipedia page is akin to
making it into a textbook about the subject area and being read by a
much wider audience that goes beyond the scientific community.

\emph{PLOS Biology} is the PLOS journal with the highest percentage of
articles recommended in F1000Prime and mentioned in Wikipedia, but there
is only partial overlap between the two groups of articles because they
focus on different audiences (\textbf{Figure 5}). These recommendations
and mentions in turn show correlations with other metrics, but not
simple ones; you can't assume, for example, that highly cited articles
are more likely to be recommended by F1000Prime, so it will be
interesting to monitor these trends now that we include this
information.

\begin{Shaded}
\begin{Highlighting}[]
\CommentTok{# code for figure 5: Venn diagram F1000 vs. Wikipedia for PLOS Biology}
\CommentTok{# articles}

\CommentTok{# load required libraries}
\KeywordTok{library}\NormalTok{(}\StringTok{"plyr"}\NormalTok{)}
\KeywordTok{library}\NormalTok{(}\StringTok{"VennDiagram"}\NormalTok{)}

\CommentTok{# load May 20, 2013 ALM report}
\NormalTok{alm <-}\StringTok{ }\KeywordTok{read.csv}\NormalTok{(}\StringTok{"../data/alm_report_plos_biology_2013-05-20.csv"}\NormalTok{, }\DataTypeTok{stringsAsFactors =} \OtherTok{FALSE}\NormalTok{)}

\CommentTok{# only look at research articles}
\NormalTok{alm <-}\StringTok{ }\KeywordTok{subset}\NormalTok{(alm, alm$article_type ==}\StringTok{ "Research Article"}\NormalTok{)}

\CommentTok{# group articles based on values in Wikipedia and F1000}
\NormalTok{reassignWikipedia <-}\StringTok{ }\NormalTok{function(x) if (x >}\StringTok{ }\DecValTok{0}\NormalTok{) }\DecValTok{1} \NormalTok{else }\DecValTok{0}
\NormalTok{alm$wikipedia_bin <-}\StringTok{ }\KeywordTok{aaply}\NormalTok{(alm$wikipedia, }\DecValTok{1}\NormalTok{, reassignWikipedia)}
\NormalTok{reassignF1000 <-}\StringTok{ }\NormalTok{function(x) if (x >}\StringTok{ }\DecValTok{0}\NormalTok{) }\DecValTok{2} \NormalTok{else }\DecValTok{0}
\NormalTok{alm$f1000_bin <-}\StringTok{ }\KeywordTok{aaply}\NormalTok{(alm$f1000, }\DecValTok{1}\NormalTok{, reassignF1000)}
\NormalTok{alm$article_group =}\StringTok{ }\NormalTok{alm$wikipedia_bin +}\StringTok{ }\NormalTok{alm$f1000_bin}
\NormalTok{reassignCombined <-}\StringTok{ }\NormalTok{function(x) if (x ==}\StringTok{ }\DecValTok{3}\NormalTok{) }\DecValTok{1} \NormalTok{else }\DecValTok{0}
\NormalTok{alm$combined_bin <-}\StringTok{ }\KeywordTok{aaply}\NormalTok{(alm$article_group, }\DecValTok{1}\NormalTok{, reassignCombined)}
\NormalTok{reassignNo <-}\StringTok{ }\NormalTok{function(x) if (x ==}\StringTok{ }\DecValTok{0}\NormalTok{) }\DecValTok{1} \NormalTok{else }\DecValTok{0}
\NormalTok{alm$no_bin <-}\StringTok{ }\KeywordTok{aaply}\NormalTok{(alm$article_group, }\DecValTok{1}\NormalTok{, reassignNo)}

\CommentTok{# remember to divide f1000_bin by 2, as this is the default value}
\NormalTok{summary <-}\StringTok{ }\KeywordTok{colSums}\NormalTok{(}\KeywordTok{subset}\NormalTok{(alm, }\DataTypeTok{select =} \KeywordTok{c}\NormalTok{(}\StringTok{"wikipedia_bin"}\NormalTok{, }\StringTok{"f1000_bin"}\NormalTok{, }\StringTok{"combined_bin"}\NormalTok{,}
    \StringTok{"no_bin"}\NormalTok{)), }\DataTypeTok{na.rm =} \OtherTok{TRUE}\NormalTok{)}
\NormalTok{rows <-}\StringTok{ }\KeywordTok{nrow}\NormalTok{(alm)}

\CommentTok{# options}
\NormalTok{plos.colors <-}\StringTok{ }\KeywordTok{c}\NormalTok{(}\StringTok{"#c9c9c7"}\NormalTok{, }\StringTok{"#0000ff"}\NormalTok{, }\StringTok{"#ff0000"}\NormalTok{)}

\CommentTok{# plot the chart}
\NormalTok{opar <-}\StringTok{ }\KeywordTok{par}\NormalTok{(}\DataTypeTok{mai =} \KeywordTok{c}\NormalTok{(}\FloatTok{0.5}\NormalTok{, }\FloatTok{0.75}\NormalTok{, }\FloatTok{3.5}\NormalTok{, }\FloatTok{0.5}\NormalTok{), }\DataTypeTok{omi =} \KeywordTok{c}\NormalTok{(}\FloatTok{0.5}\NormalTok{, }\FloatTok{0.5}\NormalTok{, }\FloatTok{1.5}\NormalTok{, }\FloatTok{0.5}\NormalTok{), }\DataTypeTok{mgp =} \KeywordTok{c}\NormalTok{(}\DecValTok{3}\NormalTok{,}
    \FloatTok{0.5}\NormalTok{, }\FloatTok{0.5}\NormalTok{), }\DataTypeTok{fg =} \StringTok{"black"}\NormalTok{, }\DataTypeTok{cex.main =} \DecValTok{2}\NormalTok{, }\DataTypeTok{cex.lab =} \FloatTok{1.5}\NormalTok{, }\DataTypeTok{col =} \NormalTok{plos.color,}
    \DataTypeTok{col.main =} \NormalTok{plos.color, }\DataTypeTok{col.lab =} \NormalTok{plos.color, }\DataTypeTok{xaxs =} \StringTok{"i"}\NormalTok{, }\DataTypeTok{yaxs =} \StringTok{"i"}\NormalTok{)}

\NormalTok{venn.plot <-}\StringTok{ }\KeywordTok{draw.triple.venn}\NormalTok{(}\DataTypeTok{area1 =} \NormalTok{rows, }\DataTypeTok{area2 =} \NormalTok{summary[}\DecValTok{1}\NormalTok{], }\DataTypeTok{area3 =} \NormalTok{summary[}\DecValTok{2}\NormalTok{]/}\DecValTok{2}\NormalTok{,}
    \DataTypeTok{n12 =} \NormalTok{summary[}\DecValTok{1}\NormalTok{], }\DataTypeTok{n23 =} \NormalTok{summary[}\DecValTok{3}\NormalTok{], }\DataTypeTok{n13 =} \NormalTok{summary[}\DecValTok{2}\NormalTok{]/}\DecValTok{2}\NormalTok{, }\DataTypeTok{n123 =} \NormalTok{summary[}\DecValTok{3}\NormalTok{],}
    \DataTypeTok{euler.d =} \OtherTok{TRUE}\NormalTok{, }\DataTypeTok{scaled =} \OtherTok{TRUE}\NormalTok{, }\DataTypeTok{fill =} \NormalTok{plos.colors, }\DataTypeTok{cex =} \DecValTok{2}\NormalTok{, }\DataTypeTok{fontfamily =} \KeywordTok{rep}\NormalTok{(}\StringTok{"sans"}\NormalTok{,}
        \DecValTok{7}\NormalTok{))}

\KeywordTok{par}\NormalTok{(opar)}
\end{Highlighting}
\end{Shaded}

\begin{figure}[htbp]
\centering
\includegraphics{/images/2013-12-11_figure_5.svg}
\caption{\textbf{Figure 5. PLOS Biology articles: sites of
recommendation and discussion.} Number of \emph{PLOS Biology} research
articles published up to May 20, 2013 that have been recommended by
F1000Prime (red) or mentioned in Wikipedia (blue).}
\end{figure}

With the increasing availability of ALM data, there comes a growing need
to provide tools that will allow the community to interrogate them. A
good first step for researchers, research administrators, and others
interested in looking at the metrics of a larger set of PLOS articles is
the recently launched ALM Reports tool {[}@ALM2013{]}. There are also a
growing number of service providers, including Altmetric.com
{[}@Altmetric2013{]}, ImpactStory {[}@Impactstory2013{]}, and Plum
Analytics {[}@Plum2013{]} that provide similar services for articles
from other publishers.

As article-level metrics become increasingly used by publishers,
funders, universities, and researchers, one of the major challenges to
overcome is ensuring that standards and best practices are widely
adopted and understood. The National Information Standards Organization
(NISO) was recently awarded a grant by the Alfred P. Sloan Foundation to
work on this {[}@NISO2013{]}, and PLOS is actively involved in this
project. We look forward to further developing our article-level metrics
and to having them adopted by other publishers, which hopefully will
pave the way to their wide incorporation into research and researcher
assessments.

\subsubsection{Supporting Information}\label{supporting-information}

\textbf{\href{http://dx.doi.org/10.1371/journal.pbio.1001687.s001}{Data
S1}. Dataset of ALM for PLOS Biology articles used in the text, and R
scripts that were used to produce figures.} The data were collected on
May 20, 2013 and include all \emph{PLOS Biology} articles published up
to that day. Data for F1000Prime were collected on August 15, 2013. All
charts were produced with R version 3.0.0.
